% Generated by GrindEQ Word-to-LaTeX 
\documentclass{article} %%% use \documentstyle for old LaTeX compilers

\usepackage[english]{babel} %%% 'french', 'german', 'spanish', 'danish', etc.
\usepackage{amssymb}
\usepackage{amsmath}
\usepackage{txfonts}
\usepackage{mathdots}
\usepackage[classicReIm]{kpfonts}
\usepackage[dvips]{graphicx} %%% use 'pdftex' instead of 'dvips' for PDF output

% You can include more LaTeX packages here 


\begin{document}

%\selectlanguage{english} %%% remove comment delimiter ('%') and select language if required


\noindent 

\noindent 

\noindent 

\noindent 

\includegraphics*[width=0.47in, height=0.63in, keepaspectratio=false]{image1}

 


\noindent \textbf{}

\noindent \textbf{\includegraphics*[width=1.08in, height=1.46in, keepaspectratio=false]{image2}}

\noindent \textbf{}

\noindent \textbf{UNIVERSIDAD PRIVADA DE TACNA}

\noindent \textbf{}

\noindent \textbf{FACULTAD DE INGENIERIA}

\noindent \textbf{\textit{}}

\noindent \textbf{Escuela Profesional de Ingenier\'{i}a de Sistemas}

\noindent \textbf{\textit{}}

\noindent \textbf{}

\noindent 

\noindent \textbf{ INFORME TRABAJO DE UNIDAD II}

\noindent \textbf{``Modelo Dimensional vs Modelo Tabular''}

\noindent \textbf{\textit{}}

\noindent \textbf{}

\noindent \textbf{ }

\noindent Curso: Inteligencia de Negocios

\noindent \textbf{\textit{}}

\noindent \textbf{}

\noindent \textbf{}

\noindent Docente: Ing. \textbf{\textit{}}

\noindent \textbf{}

\noindent \textbf{}

\noindent \textbf{Jhosmell Gyno Alfaro Musaja~ ~ ~ ~ ~(2015053223)}

\noindent \textbf{Jhon Peter Aguilar Atencio~ ~ ~ ~ ~ ~ ~(2015053222)}

\noindent \textbf{Guimer Coquira Coaquira~ ~ ~ ~ ~ ~ ~ ~(2015053226)}

\noindent \textbf{Nilton Edy P\'{e}rez Mamani~ ~ ~ ~ ~ ~ ~ ~ (2015053233)}

\noindent 

\noindent 

\noindent \textbf{Tacna -- Per\'{u}}
\[2019\]
\\
\\
\\
\\
\\
\\
\\
\\
\\
\\
\\
\\
\\
\\
\\
\textbf{}

\noindent \textbf{}

\noindent \textbf{INDICE}

\noindent 

\noindent
\begin{center}
\begin{flushleft}
I.	INFORMACIÓN GENERAL......................3\\
OBJETIVOS:..................................................3\\
EQUIPOS, MATERIALES, PROGRAMAS Y RECURSOS UTILIZADOs:.......3\\
II.	MARCO TEORICO...........................................3\\
III.	PROCEDIMIENTO.......................................3\\
IV.	CUESTIONARIO............................................4\\
CONCLUSIONES................................................4\\
RECOMENDACIONES.............................................5\\
WEBGRAFIA...................................................6\\
\end{flushleft}
\end{center}

\noindent \textbf{\eject }

\noindent \textbf{\underbar{``MODELO DIMENSIONAL VS MODELO TABULAR''}}

\noindent \textbf{\textit{}}

\noindent 

\begin{enumerate}
\item  \textbf{INFORMACI\'{O}N GENERAL}
\end{enumerate}

\noindent 

\noindent 
\section{OBJETIVOS:}

\noindent 

\begin{enumerate}
\item  Estudio de Diferencias entre Modelo Dimensional vs Modelo Tabular
\end{enumerate}

\noindent 

\noindent 
\section{EQUIPOS, MATERIALES, PROGRAMAS Y RECURSOS UTILIZADOS:}

\noindent 

\begin{enumerate}
\item  Computadora con sistema operativo Windows 10. 
\end{enumerate}

\noindent 

\begin{enumerate}
\item  \textbf{MARCO TEORICO}
\end{enumerate}

\noindent 

\noindent \textbf{\underbar{MODELO DIMENSIONAL}}

\noindent \textbf{\underbar{}}

\noindent B\'{a}sicamente el Modelo Dimensional es el nombre que se le da a una t\'{e}cnica utilizada especialmente en Data Warehouses. Este modelo difiere bastante del modelo Entidad-Relaci\'{o}n que normalmente conocemos.

\noindent 

\noindent El Modelo Dimensional busca presentar la informaci\'{o}n de una manera est\'{a}ndar, sencilla y sobre todo intuitiva para los usuarios, adem\'{a}s de que permite accesos a la informaci\'{o}n mucho m\'{a}s r\'{a}pida por parte de los manejadores de bases de datos.

\noindent 

\noindent Cada Modelo Dimensional est\'{a} compuesto por una tabla llamada ``de hechos'' y por un conjunto de peque\~{n}as tablas llamadas ``dimensiones ``. Cada dimensi\'{o}n contiene una llave primaria que se ``conecta'' a la tabla de hechos manteniendo una relaci\'{o}n de 1 a muchos).

\noindent 

\noindent 

\noindent 

\begin{enumerate}
\item  \textbf{Modelo dimensional o multidimensional}
\end{enumerate}

\noindent \textbf{}

\noindent El modelo multidimensional dentro del entorno de las bases de datos es una disciplina de dise\~{n}o que se sustenta en el modelo entidad-relaci\'{o}n y en las realidades de la ingenier\'{i}a de texto y datos num\'{e}ricos. 

\noindent 

\noindent Modela las particularidades de los procesos que ocurren en una organizaci\'{o}n, dividi\'{e}ndolos en mediciones y entorno. Las medidas son en su mayor\'{i}a, medidas num\'{e}ricas, y se les denomina hechos. Alrededor de estos hechos existe un contexto que describe en qu\'{e} condiciones y en qu\'{e} momento se registr\'{o} este hecho. Aunque el entorno se ve como un todo, existen registros l\'{o}gicos de diferentes caracter\'{i}sticas que describen un hecho, por ejemplo, si el hecho referido, es la venta de un producto en una cadena de tiendas, se podr\'{i}a dividir el entorno que rodea al hecho de la cantidad vendida, en el producto vendido, el cliente que lo compr\'{o}, la tienda y la fecha en que se realiz\'{o} la venta. A estas divisiones se le denomina dimensiones y a diferencia de los hechos que son num\'{e}ricos, estos son fundamentalmente textos descriptivos.

\noindent 

\noindent Las medidas, como se expres\'{o} anteriormente, se registran en las tablas de hechos, siendo la llave de esta tabla, la combinaci\'{o}n de las m\'{u}ltiples llaves for\'{a}neas que hacen referencia   a las dimensiones que describen la ocurrencia de este hecho, en otras palabras, cada una de las llaves extranjeras en las tablas de hecho se corresponden con la llave primaria de una dimensi\'{o}n. 

\noindent \includegraphics*[width=3.09in, height=2.40in, keepaspectratio=false]{image3}\includegraphics*[width=3.21in, height=2.39in, keepaspectratio=false]{image4}

\noindent 

\begin{enumerate}
\item  \textbf{Proceso de negocio}
\end{enumerate}

\noindent 

\noindent Un proceso de negocio es un conjunto de tareas relacionadas l\'{o}gicamente llevadas a cabo para lograr un resultado de negocio definido. Cada proceso de negocio tiene sus entradas, funciones y salidas. Las entradas son requisitos que deben tenerse antes de que una funci\'{o}n pueda ser aplicada. Cuando una funci\'{o}n es aplicada a las entradas de un m\'{e}todo, tendremos ciertas salidas resultantes.

\noindent 

\noindent Es una colecci\'{o}n de actividades estructurales relacionadas que producen un valor para la organizaci\'{o}n, sus inversores o sus clientes. Es, por ejemplo, el proceso a trav\'{e}s del que una organizaci\'{o}n ofrece sus servicios a sus clientes.

\noindent 

\noindent Un proceso de negocio puede ser parte de un proceso mayor que lo abarque o bien puede incluir otros procesos de negocio que deban ser incluidos en su funci\'{o}n. En este contexto un proceso de negocio puede ser visto a varios niveles de granularidad. El enlace entre procesos de negocio y generaci\'{o}n de valor lleva a algunos practicantes a ver los procesos de negocio como los flujos de trabajo que efect\'{u}an las tareas de una organizaci\'{o}n. 

\noindent 

\noindent Los procesos poseen las siguientes caracter\'{i}sticas:

\noindent 

\begin{enumerate}
\item  Pueden ser medidos y est\'{a}n orientados al rendimiento

\item  Tienen resultados espec\'{i}ficos

\item  Entregan resultados a clientes o ``stakeholders''

\item  Responden a alguna acci\'{o}n o evento espec\'{i}fico

\item  Las actividades deben agregar valor a las entradas del proceso.
\end{enumerate}

\noindent 

\noindent Los procesos de negocio pueden ser vistos como un recetario para hacer funcionar un negocio y alcanzar las metas definidas en la estrategia de negocio de la empresa. Las dos formas principales de visualizar una organizaci\'{o}n son la vista funcional y la vista de procesos.

\noindent 

\noindent Hay tres tipos de procesos de negocio:

\noindent 

\begin{enumerate}
\item  Procesos estrat\'{e}gicos -- Estos procesos dan orientaci\'{o}n al negocio. Por ejemplo, ``Planeacion estrategica'', ``Establecer objetivos y metas''.

\item  Procesos sustantivos-- Estos procesos dan el valor al cliente, son la parte principal del negocio. Por ejemplo, ``Repartir mercanc\'{i}as''.

\item  \includegraphics*[width=4.21in, height=2.54in, keepaspectratio=false]{image5}Procesos de apoyo vertical u horizontal -- Estos procesos dan soporte a los procesos centrales. Por ejemplo, ``Registrar los hechos econ\'{o}micos'', ``Dar Soporte/Servicio t\'{e}cnico''.
\end{enumerate}

\noindent 

\begin{enumerate}
\item  \textbf{Medida}
\end{enumerate}

\noindent \textbf{}

\noindent Las medidas m\'{a}s \'{u}tiles para incluir en una tabla de hechos son los aditivos, es decir, aquellas medidas que pueden ser sumadas como por ejemplo la cantidad de producto vendido, los costes de producci\'{o}n o el dinero obtenido por las ventas; son medidas num\'{e}ricas que pueden calcularse con la suma de varias cantidades de la tabla. En consecuencia, por lo general los hechos a almacenar en una tabla de hechos van a ser casi siempre valores num\'{e}ricos, enteros o reales.

\noindent \includegraphics*[width=3.41in, height=2.26in, keepaspectratio=false]{image6}

\begin{enumerate}
\item  \textbf{Tabla dimensional o dimensi\'{o}n}
\end{enumerate}

\noindent 

\noindent En un almac\'{e}n de datos o un sistema OLAP, la construcci\'{o}n de Cubos OLAP requiere de una tabla de hechos y varias tablas de dimensiones, \'{e}stas acompa\~{n}an a la tabla de hechos y determinan los par\'{a}metros (dimensiones) de los que dependen los hechos registrados en la tabla de hechos.

\noindent 

\noindent En la construcci\'{o}n de cubos OLAP, las tablas de dimensiones son elementos que contienen atributos (o campos) que se utilizan para restringir y agrupar los datos almacenados en una tabla de hechos cuando se realizan consultas sobre dichos datos en un entorno de almac\'{e}n de datos o data mart.

\noindent 

\noindent Estos datos sobre dimensiones son par\'{a}metros de los que dependen otros datos que ser\'{a}n objeto de estudio y an\'{a}lisis y que est\'{a}n contenidos en la tabla de hechos. Las tablas de dimensiones ayudan a realizar ese estudio/an\'{a}lisis aportando informaci\'{o}n sobre los datos de la tabla de hechos, por lo que puede decirse que en un cubo OLAP, la tabla de hechos contiene los datos de inter\'{e}s y las tablas de dimensiones contienen metadatos sobre dichos hechos.

\noindent \includegraphics*[width=3.28in, height=2.70in, keepaspectratio=false]{image7}

\noindent 

\noindent 

\begin{enumerate}
\item  \textbf{Bases de datos multidimensionales}
\end{enumerate}

\noindent 

\noindent Las bases de datos multidimensionales constituyen la forma de almacenamiento de los Datawerehouse porque facilitan una forma flexible de acceso a los datos almacenados, los cuales se van a localizar en diferentes dimensiones, y se visualizan como un cubo multidimensional, en donde las variables asociadas existen a lo largo de varios ejes o dimensiones, y la intersecci\'{o}n de las mismas representa la medida, indicador o el hecho que se est\'{a} evaluando.

\noindent 

\noindent Las bases de datos multidimensionales implican tres variantes posibles de modelamiento, que permiten realizar consultas de soporte de decisi\'{o}n:

\noindent 

\begin{enumerate}
\item  Esquema en estrella.

\item  Esquema copo de nieve.

\item  Esquema constelaci\'{o}n.
\end{enumerate}

\noindent 

\begin{enumerate}
\item  \textbf{Tabla de hechos}
\end{enumerate}

\noindent 

\noindent Las tablas de hechos representan los procesos que ocurren en la organizaci\'{o}n, son independientes entre s\'{i} (no se relacionan unas con otras). En estas, se almacenan las medidas num\'{e}ricas de la organizaci\'{o}n. Cada medida, se corresponde con una intersecci\'{o}n de valores de las dimensiones y generalmente se trata de cantidades num\'{e}ricas, continuamente evaluadas y aditivas. La raz\'{o}n de estas caracter\'{i}sticas es que facilita que los miles de registros que involucran una consulta sean comprimidos m\'{a}s f\'{a}cilmente y se pueda dar respuesta con rapidez, a una solicitud que abarque gran cantidad de informaci\'{o}n.

\noindent 

\noindent La llave de la tabla de hechos es una llave compuesta, debido a que se forma de la composici\'{o}n de las llaves primarias de las tablas dimensionales a las que est\'{a} unida. se pueden distinguir dos tipos de columnas en una tabla de hechos, columnas de hechos y columnas llaves. Las columnas de hechos almacenan las medidas del negocio que se quieren controlar y las columnas llaves, forman parte de la llave de la tabla.

\noindent 

\noindent \includegraphics*[width=4.69in, height=3.40in, keepaspectratio=false]{image8}Existen tablas de hechos que no contienen medidas, a estas tablas se les denomina tablas de hechos sin hechos. La sem\'{a}ntica de la relaci\'{o}n entre las dimensiones que definen la llave de esta tabla de hechos implica por s\'{i} sola la ocurrencia de un evento, por ejemplo, si se quiere representar el hecho de que un estudiante matricul\'{o} en una universidad, la combinaci\'{o}n de las siguientes dimensiones definir\'{i}a este suceso: el estudiante matriculado, la carrera en que matricul\'{o}, la fecha de matr\'{i}cula, el tipo de curso que va a cursar, etc\'{e}tera.

\noindent 

\noindent 

\noindent 

\noindent \textbf{\underbar{MODELO TABULAR}}

\noindent \textbf{\underbar{}}

\noindent En t\'{e}rminos muy simples un modelo tabular es una base de datos OLAP cuyo almacenamiento est\'{a} en memoria RAM. Debido a su enfoque (similar a una base de datos columnar) alcanza altos ratios de compresi\'{o}n gestionando gran cantidad de informaci\'{o}n en poca memoria. Debido a que est\'{a} en memoria ofrece un r\'{a}pido acceso a la informaci\'{o}n.

\noindent 

\begin{enumerate}
\item  \textbf{Ventajas}
\end{enumerate}

\noindent Al implementar un modelo tabular se crea una base de datos del modelo en un entorno de pruebas, ensayo o producci\'{o}n. Los usuarios pueden conectarse al modelo implementado mediante un archivo de conexi\'{o}n. bism en Sharepoint o mediante una conexi\'{o}n de datos directamente desde las aplicaciones cliente de informes, como Microsoft Excel, Power View, o una aplicaci\'{o}n personalizada. La base de datos del \'{a}rea de trabajo del modelo, que se genera al crear un proyecto de modelo tabular en SQL Server Data Tools (SSDT)y se usa para crear el modelo, permanecer\'{a} en la instancia del servidor del \'{a}rea de trabajo, lo que permite realizar cambios en el proyecto de modelo y volver a implementarlo despu\'{e}s en el entorno de pruebas, ensayo o producci\'{o}n cuando sea necesario.

\noindent 

\begin{enumerate}
\item  \textbf{Implementar un modelo tabular de SQL Server Data Tools (SSDT)}
\end{enumerate}

\noindent \textbf{}

\noindent La implementaci\'{o}n es un proceso sencillo; sin embargo, se deben realizar algunos pasos para asegurarse de que el modelo se implementa en la instancia adecuada de Analysis Services y con las opciones de configuraci\'{o}n correctas.

\noindent 

\noindent Los modelos tabulares se definen con varias propiedades espec\'{i}ficas de la implementaci\'{o}n. Durante la implementaci\'{o}n, se establece una conexi\'{o}n con la instancia de Analysis Services especificada en la propiedad Servidor. 

\noindent 

\noindent A continuaci\'{o}n, se crea en esa instancia una nueva base de datos modelo con el nombre especificado en la propiedad Database, si no existe ninguna. Los metadatos del archivo Model.bim del proyecto de modelo se usan para configurar los objetos de la base de datos de modelo en el servidor de implementaci\'{o}n. La Opci\'{o}n de procesamientole permite especificar si solo se implementan los metadatos del modelo, si se crea la base de datos del modelo o, si se especifica Predeterminado o Completo, las credenciales de suplantaci\'{o}n usadas para conectarse con or\'{i}genes de datos se pasan "en memoria" de la base de datos del \'{a}rea de trabajo del modelo a la base de datos implementada del modelo. A continuaci\'{o}n, Analysis Services ejecuta el procesamiento para rellenar los datos en el modelo implementado. Una vez completado el proceso de implementaci\'{o}n, las aplicaciones cliente pueden conectarse con el modelo mediante una conexi\'{o}n de datos o mediante un archivo de conexi\'{o}n.

\noindent 

\noindent 

\noindent 

\noindent \includegraphics*[width=4.72in, height=2.97in, keepaspectratio=false, trim=0.72in 0.86in 0.90in 0.92in]{image9}

\noindent 

\noindent 

\noindent 

\noindent 

\noindent 

\noindent 

\noindent 

\noindent 

\noindent \textbf{}

\noindent \textbf{}

\noindent \textbf{}

\noindent \textbf{}

\noindent \textbf{}

\noindent \textbf{}

\noindent \textbf{}

\noindent \textbf{}

\noindent \textbf{}

\noindent \textbf{}

\noindent \textbf{}

\noindent 

\noindent 

\begin{enumerate}
\item  \textbf{CUESTIONARIO} 
\end{enumerate}

\noindent 

\noindent \textbf{¿PREGUNTAS COMUNES?}

\noindent 

\begin{enumerate}
\item  El Tabular Model sustituye el modelo multidimensional o solo es una mejora de Power Pivot.

\item  Deber\'{i}a desarrollar solo modelos Tabulares.

\item  Quien es m\'{a}s r\'{a}pido los modelos en memoria o los MOLAP.

\item  El desarrollo en Tabular Model es m\'{a}s sencillo. Pero solo conocemos de Dimensiones, Hechos y MDX.
\end{enumerate}

\noindent 

\noindent \includegraphics*[width=4.95in, height=2.10in, keepaspectratio=false]{image10}

\noindent 

\noindent 

\noindent 

\noindent 

\noindent 

\noindent 

\noindent 

\noindent 

\noindent 

\noindent 

\noindent 

\noindent 

\noindent 

\noindent 

\noindent 

\noindent 
\section{CONCLUSIONES}

\noindent 

\begin{enumerate}
\item  El modelo dimensional y el novedoso modelo tabular, as\'{i} como en las funcionalidades de herramientas de visualizaci\'{o}n como Microsoft Excel y Power View, en funci\'{o}n de agilizar y enriquecer el ambiente anal\'{i}tico puesto a disposici\'{o}n de los especialistas y ejecutivos. Se constat\'{o} que el modelo tabular no constituye un aporte conceptualmente diferente, sino una implementaci\'{o}n alternativa del modelo dimensional de datos para la herramienta de procesamiento anal\'{i}tico SQL Server 2012 Analysis Services, aun cuando el soporte sobre el almacenamiento columnar y las bases de datos in-memory le aportan un inter\'{e}s actual y una perspectiva fruct\'{i}fera al modelo tabular, m\'{a}s all\'{a} de lo que puede ofrecer Microsoft.
\end{enumerate}

\noindent 

\begin{enumerate}
\item  A partir del estudio de los modelos multidimensional y tabular, la experiencia adquirida en el desarrollo del sistema y el comportamiento de los resultados obtenidos en los experimentos llevados a cabo, se pudo realizar una comparaci\'{o}n cualitativa y pr\'{a}ctica de ambas opciones. Como resultado, se esbozaron consideraciones generales en cuanto a fortalezas y debilidades contribuyendo al trabajo ulterior en esta \'{a}rea del conocimiento. Se concluye que el modelo tabular ofrece una alternativa de modelaci\'{o}n con la que los usuarios se identifican mejor a partir de sus necesidades de an\'{a}lisis y consultas, pero se basa en el uso y el aprovechamiento de elevados recursos de hardware, principalmente memoria RAM. Por otro lado, se corrobor\'{o} que el modelo multidimensional puede constituir la mejor opci\'{o}n si se requiere una compleja modelaci\'{o}n de la l\'{o}gica del negocio o se necesita una soluci\'{o}n escalable y que maneje grandes vol\'{u}menes de datos.
\end{enumerate}

\noindent 

\noindent 

\noindent 

\noindent 

\noindent 

\noindent 

\noindent 

\noindent 

\noindent 

\noindent 
\section{RECOMENDACIONES}

\begin{enumerate}
\item \textbf{ }Para validar la soluci\'{o}n propuesta, se dise\~{n}aron cuatro experimentos que permiten corroborar algunos de los supuestos te\'{o}ricos a los cuales se ha arribado en la presente investigaci\'{o}n. 
\end{enumerate}

\noindent 

\begin{enumerate}
\item  Estos experimentos responden a las fases principales en la implementaci\'{o}n y la presentaci\'{o}n de los resultados. Debido al gran volumen de datos existente en los sistemas fuentes, correspondientes a los \'{u}ltimos tres a\~{n}os, en el escenario comercial se utilizaron datos de tres de las principales sucursales de 

\item  CIMEX en el pa\'{i}s, a saber:

\item  Pinar del R\'{i}o, Holgu\'{i}n y La Habana. En el escenario contable se emplearon los datos de todas las entidades de CIMEX.

\item  Los experimentos dise\~{n}ados se ejecutaron en un servidor con sistema operativo Windows Server 2008 R2 Standard (Service Pack 1), un procesador DualCore AMD Opteron(tm) Processor 2218 de velocidad 2.60 GHz, con memoria RAM de 3.00 GB y una arquitectura de 64 bit.
\end{enumerate}

\noindent 

\noindent 

\noindent 

\noindent 

\noindent 

\noindent 

\noindent 

\noindent 

\noindent 

\noindent 

\noindent 

\noindent 

\noindent 

\noindent 

\noindent 

\noindent 

\noindent 

\noindent 
\section{WEBGRAFIA}

\noindent 

\begin{enumerate}
\item  \'{A}1.  ALBANO, A., ROSA, L. D., DUMITRESCU, C., GOGLIA, L., GOGLIA, R.  AND MINEI, V. Another Example of a Data Warehouse System Based on Transposed Files. En 10th International Conference on Advances in Database Technology: Springer-Verlag, 2006, p. 1110-1114. 2.  

\item  BOATENG, O., SINGH, J., GREESHMA AND SINGH, P. DATA WAREHOUSING. Business Intelligence Journal, 2012, vol. 5 (no. 2), pp. 224-234. 3.  

\item  STONEBRAKER, M., ABADI, D. J., BATKIN, A. AND CHEN, X.  C-Store: A Column-oriented DBMS En 31st Very Large Data Bases (VLDB) Conference, Trondheim, Norway, 2005. 4.

\item  RUSSO, M., FERRARI, A.  AND WEBB, C. Microsoft SQL Server 2012 Analysis Services: The BISM Tabular Model. Microsoft Press, 2012. 5.  

\item  GARC\'{I}A, L., SIM\'{O}N, A., TORRES, M. AND ESPINOSA, Y. Soluci\'{o}n de inteligencia de negocios para la integraci\'{o}n de la informaci\'{o}n comercial y contable.  En Congreso Internacional de Matem\'{a}tica y Computaci\'{o}n 2013, La Habana, Cuba, 2013. 6.  EVELSON, B. AND NICOLSON, N. Topic Overview: Business Intelligence.  2008. Disponible en Internet:$\mathrm{<}$http://bit.ly/foresterbi$\mathrm{>}$.  7.  BERNABEU, R. D. HEFESTO. Data Warehousing: Investigaci\'{o}n y Sistematizaci\'{o}n de Conceptos. Hefesto: Metodolog\'{i}a para la Construcci\'{o}n de un Data Warehouse. C\'{o}rdoba, Argentina, 2010. 8.  

\item  GARC\'{I}A HERN\'{A}NDEZ, L., OLIVA SANTOS, R., PRENDES ARENCIBIA, H., VEL\'{A}ZQUEZ VIDAL, L. AND V\'{E}LIZ MONTEAGUDO, M. La Inteligencia de Negocios desde la perspectiva de los datos. En: 4to Evento Nacional de Inform\'{a}tica/08, Cuba, 2008. 9.  PENDSE, N. OLAP Architectures. En: The OLAP Report, 2005. 10.  

\item  VEL\'{A}ZQUEZ VIDAL, L.  Herramienta    gen\'{e}rica para la poblaci\'{o}n del Data Warehouse Informacional. Tesis en opci\'{o}n del t\'{i}tulo de M\'{a}ster en Ciencia de la Computaci\'{o}n. Facultad de Matem\'{a}tica y Computaci\'{o}n. Universidad de La Habana, 2009. 11.  
\end{enumerate}

\noindent \textbf{}


\end{document}
